\documentclass[%
	%draft,
	%submission,
	%compressed,
	final,
	%
	%technote,
	%internal,
	%submitted,
	%inpress,
	reprint,
	%
	%titlepage,
	notitlepage,
	%anonymous,
	narroweqnarray,
	inline,
	twoside,
        invited,
	]{ieee}

\newcommand{\latexiie}{\LaTeX2{\Large$_\varepsilon$}}

%\usepackage{ieeetsp}	% if you want the "trans. sig. pro." style
%\usepackage{ieeetc}	% if you want the "trans. comp." style
%\usepackage{ieeeimtc}	% if you want the IMTC conference style

% Use the `endfloat' package to move figures and tables to the end
% of the paper. Useful for `submission' mode.
%\usepackage {endfloat}

% Use the `times' package to use Helvetica and Times-Roman fonts
% instead of the standard Computer Modern fonts. Useful for the 
% IEEE Computer Society transactions.
%\usepackage{times}
% (Note: If you have the commercial package `mathtime,' (from 
% y&y (http://www.yandy.com), it is much better, but the `times' 
% package works too). So, if you have it...
%\usepackage {mathtime}

% for any plug-in code... insert it here. For example, the CDC style...
%\usepackage{ieeecdc}

\begin{document}

%----------------------------------------------------------------------
% Title Information, Abstract and Keywords
%----------------------------------------------------------------------
\title[Desarrollos de Investigacion a nivel de Hardware en Argentina]{Desarrollos de Investigacion a nivel de Hardware en Argentina}

% format author this way for journal articles.
% MAKE SURE THERE ARE NO SPACES BEFORE A \member OR \authorinfo
% COMMAND (this also means `don't break the line before these
% commands).
\author[]{KILMURRAY, Gerardo Luis\authorinfo{gerakilmurray@gmail.com}
- PICCO, Gonzalo Martin\authorinfo{gonzalopicco@gmail.com} 
}

% \titletext{Estudiantes de computaci\'on Universidad Nacional de R\'io Cuarto}
\lognumber{version 1}
% \pubitemident{S 0018--9456(97)09426--6}
\loginfo{Estudiantes de computaci\'on Universidad Nacional de R\'io Cuarto}
\firstpage{1}

\confplacedate{Ottawa, Canada, May 19--21, 1997}

\maketitle               

\begin{abstract} 
Our premise is that a researcher should be able to use his or her time
doing research, and not fighting with a text formatter. Modern
formatters have macro capability. If the proper macros are written,
the text formatting for all IEEE publications may be accomplished
automatically.  We propose that proponents of various text-processing
systems write macro packages for their own systems.  The scientific
community would benefit greatly.  One example is provided for users of
the \latexiie\ system. It is available at
\mbox{http://www--isl.stanford.edu/ieee/} or at the ftp site
\mbox{ftp://isl.stanford.edu/pub/ieee/}. A sim\-i\-lar macro package
is being developed for users of Microsoft Word, and can be found at
the world-wide-web addresses given in the Conclusion.
\end{abstract}

% \begin{keywords}
% \end{keywords}

%----------------------------------------------------------------------
% SECTION I: Introduction
%----------------------------------------------------------------------
\section{Introduction}


\begin{thebibliography}{1}

\bibitem{lamport}
Leslie Lamport,
\newblock {\em A Document Preparation System: {\LaTeX} User's Guide and
  Reference Manual},
\newblock Addison-Wesley, Reading, MA, 2nd edition, 1994.
\newblock Be sure to get the updated version for \latexiie!

\bibitem{goossens}
Michel Goossens, Frank Mittelbach, and Alexander Samarin,
\newblock {\em The {\LaTeX} Companion},
\newblock Addison-Wesley, Reading, MA, 1994.

\end{thebibliography}

%----------------------------------------------------------------------

\begin{biography}{Gregory L. Plett} 
(S'97) was born in Ottawa, ON, in 1968. He received the B.Eng.\ degree
in computer systems engineering with high distinction from Carleton
University, Ottawa, in 1990, and the M.S.\ degree in electrical
engineering from Stanford University, CA, in 1992.  He is currently a
Ph.D.\ candidate at Stanford University, where he is researching
aspects of adaptive control under the supervision of Professor Bernard
Widrow.
\end{biography}


\begin{biography}{Istv\'{a}n Koll\'{a}r} 
(M'87--SM'93--F'97) was born in Budapest, Hungary, in 1954. He graduated 
in electrical engineering from the Technical University of Budapest in 
1977 and in 1985 received the degree ``Candidate of Sciences'' (the 
equivalent of Ph.D.) from the Hungarian Academy of Sciences, and the 
degree dr.tech.\ from the Technical University of Budapest.

From September 1993 to June 1995, he was a Fulbright Scholar and
visiting associate professor in the Department of Electrical
Engineering, Stanford University. He is professor of electrical
engineering, Department of Measurement and Information Systems,
Technical University of Budapest. His research interests span the
areas of digital and analog signal processing, measurement theory, and
system identification. He has published about 50 scientific papers and
is coauthor of the book \emph{Technology of Electrical Measurements},
(L.\ Schnell, ed., Wiley, 1993). He authored the \emph{Frequency
Domain System Identification Toolbox} for Matlab.
\end{biography}

\end{document}


