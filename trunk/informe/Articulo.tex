\documentclass[%
	draft,
	%submission,
	%compressed,
	%final,
	%
	%technote,
	%internal,
	%submitted,
	%inpress,
%	reprint,
%
	%titlepage,
	notitlepage,
	%anonymous,
	narroweqnarray,
	inline,
	twoside,
%       invited,
	]{ieee}

\newcommand{\latexiie}{\LaTeX2{\Large$_\varepsilon$}}

\usepackage[spanish]{babel}

\begin{document}

%----------------------------------------------------------------------
% Title Information, Abstract and Keywords
%----------------------------------------------------------------------
\title[Desarrollos de Investigacion a nivel de Hardware en Argentina]{Desarrollos de Investigacion a nivel de Hardware en Argentina}

% format author this way for journal articles.
% MAKE SURE THERE ARE NO SPACES BEFORE A \member OR \authorinfo
% COMMAND (this also means `don't break the line before these
% commands).
 \author[]{KILMURRAY, Gerardo Luis %\authorinfo{gerakilmurray@gmail.com}
 - PICCO, Gonzalo Martin%\authorinfo{gonzalopicco@gmail.com} 
 }
\journal{Monograf\'ia Historia de la Inform\'atica}
% \titletext{Estudiantes de computaci\'on Universidad Nacional de R\'io Cuarto}
% \lognumber{version 1}
% \pubitemident{S 0018--9456(97)09426--6}
\loginfo{Estudiantes de computaci\'on Universidad Nacional de R\'io Cuarto}
\firstpage{1}

% \confplacedate{Ottawa, Canada, May 19--21, 1997}

\maketitle 
\sloppy

\begin{abstract} 
Argentina ha pasado por millones de avances y cuestiones problematicas que prodijeron que las investigaciones siempre se vieran salteadas por alguna guerra politica o golpe militar. A medida que los gobiernos radicales se adueñaban del pais, el mismo se veia agarrado por un mundo politico.
\end{abstract}

% \begin{keywords}
% \end{keywords}

%----------------------------------------------------------------------
% SECTION: Introduction
%----------------------------------------------------------------------
\section{Introduction}
Las secciones despues veremos que tal.

%----------------------------------------------------------------------
% SECTION: Estado del arte
%----------------------------------------------------------------------
\section{Estado del arte}
Las secciones despues veremos que tal

%----------------------------------------------------------------------
% SECTION: Desarrollos
%----------------------------------------------------------------------
\section{Desarrollos destacados}
Las secciones despues veremos que tal

%----------------------------------------------------------------------
% SECTION: Conclucion
%----------------------------------------------------------------------
\section{Comcluci\'on}
Las secciones despues veremos que tal


\begin{thebibliography}{1}

\bibitem{lamport}
Leslie Lamport,
\newblock {\em A Document Preparation System: {\LaTeX} User's Guide and
  Reference Manual},
\newblock Addison-Wesley, Reading, MA, 2nd edition, 1994.
\newblock Be sure to get the updated version for \latexiie!

\bibitem{goossens}
Michel Goossens, Frank Mittelbach, and Alexander Samarin,
\newblock {\em The {\LaTeX} Companion},
\newblock Addison-Wesley, Reading, MA, 1994.

\end{thebibliography}

%----------------------------------------------------------------------

\begin{biography}{Kilmurray, Gerardo Luis} 
NO se sabe despues veremos.
\end{biography}


\begin{biography}{Picco, Gonzalo Mart\'in} 
Despues veremos
\end{biography}

\end{document}